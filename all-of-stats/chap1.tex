\documentclass{article}
\usepackage{amsmath, amsfonts, amssymb}
\newcommand{\son}{\mathbb{N}}

\begin{document}
\begin{enumerate}
\item[Problem 3] Given that $\Omega$ is a sample space and $A_1, A_2, \ldots$ are
events. Let
\begin{eqnarray*}
B_n &=& \bigcup_{k=n}^\infty A_k \\
C_n &=& \bigcap_{k=n}^\infty A_k
\end{eqnarray*}

Clearly, $B_{n+1} \cup A_{n} = B_{n}$ so that $B_n \supseteq B_{n+1}$ for all $n
\ge 1$, or $B_1 \supseteq B_2 \supseteq \ldots$. Likewise $C_{n+1} \cap A_{n} = 
C_n$ so that $C_{n+1} \supseteq C_n$ or that $C_1 \subseteq C_2 \subseteq \ldots$.

Now,
\[
\omega \in \cap_{k=1}^\infty B_k \Rightarrow \omega \in B_k \forall k \in \son
\Rightarrow \omega \in \bigcup_{j=k}^\infty A_j \forall k \in \son \Rightarrow
\omega \in A_j \forall j \in \son
\]
Conversely, suppose that $\omega \in A_k$, for all $k \in \son$. Then $\omega \in
B_n$ for all $n$ and hence $\omega \in \cap_{n=1}^\infty B_n$.

On the other hand
\[
\omega \in \cup_{k=1}^\infty C_k \Rightarrow \exists k \in \son, \omega \in C_k
\Rightarrow \omega \in \cap_{j=k}^n A_j \Rightarrow \omega \in A_j \forall j
\ge k.
\]
Conversely, if $\omega \in A_k$ for all $k \ge N$ then $\omega \in C_k$ for all
$k \ge N$ and hence $\omega \in \cup_{k=1}^\infty C_k$.

\item[Problem 4] Let $I$ be an arbitrary index set and let
\[
x \in \left(\bigcup_{i \in I}A_i\right)^c \Rightarrow x \notin \bigcup_{i \in I}A_i
\Rightarrow x \notin A_i \forall i \Rightarrow x \in A_i^c \forall i \Rightarrow 
x \in \bigcap_{i\in I}A_i^c,
\]
so that
\begin{equation}\label{e1}
\left(\bigcup_{i \in I}A_i\right)^c \subset \bigcap_{i\in I}A_i^c.
\end{equation}
If,
\[
x \in \cap_{i\in I}A_i^c \Rightarrow x \in A_i^c \forall i \Rightarrow
x \notin A_i \forall i \Rightarrow x \notin \bigcup_{i \in I}A_i
\Rightarrow x \in \left(\bigcup_{i \in I}A_i\right)^c,
\]
so that
\begin{equation}\label{e2}
\bigcap_{i\in I}A_i^c \subseteq \left(\bigcup_{i \in I}A_i\right)^c.
\end{equation}
From \eqref{e1} and \eqref{e2},
\[
\bigcap_{i\in I}A_i^c = \left(\bigcup_{i \in I}A_i\right)^c.
\]

If we make the substitution $A_i \rightarrow A_i^c$ in the above equation, we 
get
\[
\bigcap_{i\in I}A_i = \left(\bigcup_{i \in I}A_i^c\right)^c \Rightarrow
\left(\bigcap_{i \in I}A_i\right)^c = \bigcup_{i \in I}A_i^c
\]

\item[Problem 5] $\Omega = \{T^i H T^j H\;|\; i \ge 0, j \ge 0\}$.

\item[Problem 6] Let $\Omega = \{0, 1, \ldots\}$. Let, if possible, there be a 
uniform probability distribution on it. That is, if $A, B$ are finite subsets of
$\Omega$ then $P(A) = P(B)$ if $|A| = |B|$. If $A$ and $B$ are singleton sets 
and if $P(A) = k$ then $P(\Omega) = \infty$. 

\item[Problem 7] Let $A_1, A_2, \ldots$ be events. We will use induction on $n$.
Let us start with the base case of $n = 2$. Then,
\[
P(\cup_{i=1}^2 A_i) = P(A_1) + P(A_2) - P(A_1 \cap A_2) \ge \sum_{i=1}^2 P(A_i).
\]
Assume that the hypothesis is true for all $n \le k$ and consider.
\[
P(\cup_{i=1}^{k+1} A_i) = P(A_{k+1} + \cup_{i=1}^k A_i) \ge P(A_{k+1}) + 
\sum_{i=1}^k P(A_i)
\]
by induction hypothesis. Therefore,
\[
P(\cup_{i=1}^{k+1} A_i) \ge \sum_{i=1}^{k+1} P(A_i)
\]
so that the hypothesis is true for all $k \in \son$.

\item[Problem 8] Let $P(B) > 0$ for a certain event $B$. Then $P(\varnothing|B)
= P(\varnothing \cap B)/P(B) = 0$; 
\[
P(A^c|B) = \frac{P(A^c \cap B}{P(B)}
\]
Since $B$ is a disjoint union of $A \cap B$ and $A^c \cap B$, $P(B) = P(A \cap B)
+ P(A^c \cap B)$ so that
\[
P(A^c|B) = 1 - \frac{P(A \cap B)}{P(B)} = 1 - P(A|B).
\]
If $A_1, A_2, \ldots$ are pairwise disjoint sets then,
\[
P(\cup_{k \ge 1}A_k | B) = \frac{P((\cup_{k\ge 1} A_k) \cap B)}{P(B)}
= \frac{P(\cup_{k\ge 1} (A_k \cap B))}{P(B)}
\]
Since the sets $A_k \cap B$ are also pairwise disjoint,
\[
P(\cup_{k \ge 1}A_k | B) = \sum_{k \ge 1}\frac{P(A_k \cap B)}{P(B)} = 
\sum_{k \ge 1}P(A_k | B).
\]
\end{enumerate}
\end{document}