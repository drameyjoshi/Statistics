\documentclass{article}
\usepackage{amsmath, amssymb, amsfonts, amsthm}
\usepackage{mathtools}
\usepackage{hyperref}
\usepackage{mathabx}

\newcommand{\son}{\mathbb{N}}
\newcommand{\soi}{\mathbb{Z}}
\newcommand{\soq}{\mathbb{Q}}
\newcommand{\sor}{\mathbb{R}}
\newcommand{\soc}{\mathbb{C}}
\newcommand{\op}{\prime}
\newcommand{\tp}{\prime\prime}
\newcommand{\td}[2]{\frac{d{#1}}{d{#2}}}
\newcommand{\bBer}{\text{Bernoulli}}

\theoremstyle{plain}
\newtheorem{thm}{Theorem}
%\numberwithin{thm}{section}

\theoremstyle{plain}
\newtheorem{prop}{Proposition}
%\numberwithin{prop}{section}

\theoremstyle{definition}
\newtheorem{defn}{Definition}
%\numberwithin{defn}{section}

\theoremstyle{remark}
\newtheorem*{rem}{Remark}

\theoremstyle{plain}
\newtheorem{cor}{Corollary}
%\numberwithin{cor}{section}

%\numberwithin{equation}
\begin{document}
\begin{prop}\label{c3p1}
Let $(\Omega, \mathcal{F}, P)$ be a probability space and $X: \Omega \rightarrow
\sor$ be a random variable. Then $\mu_F:\mathcal{F} \rightarrow [0, 1]$, where
$\mu_F((a, b)) = F(b) - F(a)$ is a probability measure on the measurable space 
$(\sor, \mathcal{B})$.
\end{prop}
\begin{proof}
We first note that $\mu_F(\sor) = 1$. Let $I_1 = (a, b)$ and $I_2 = (c, d)$ be
disjoint intervals and $I= I_1 \cup I_2$. Now, the probability of $X$ being in
$I$ is just the sum of probabilities in $I_1$ and $I_2$. Therefore $\mu_F(I) =
P(X \in I) = P(X \in I_1) + P(X \in I_2) = F(b) - F(a) + F(d) - F(c) =\mu_F(I_1)
+ \mu_F(I_2)$. Thus, $\mu_F$ is indeed a probability measure.
\end{proof}
\begin{rem}
Since $(\sor, \mathcal{B}, \mu_F)$ is a probability space, we can define 
integrals like $\int_A g(x)d\mu_F$ as Lebesgue integrals. It is common to 
express this integral as $\int_A gdF$.
\end{rem}

One can also interpret an integral $\int gdF$ as $\int g(x)f(x)dx$ as $f=F^\op$,
that is a Riemann-Stieltje integral. However, this depends on $F$ being 
differentiable. Instead of getting mired in the measure theoretic issues, we 
will settle with the Riemann-Steiltje interpretation and move on.

\begin{defn}\label{c3d1}
The expected value of a random variable is defined to be
\[
E(X) = \int xdF = \begin{cases}
\sum_x xf(x) \text{ if } X \text{ is discrete} \\
\int xf(x)dx \text{ if } X \text{ is continuous.}
\end{cases}
\]
$E(X)$ is also denoted as $\mu_X$.
\end{defn}

\begin{rem}
If $X$ is discrete then $F$ is a (normalised) counting measure. It is the 
probability of $P(X=x)$, which is the ratio of count of the events in $\Omega$
for which $X = x$ divided by the count of all events. If $p_k$ denotes $P(X=k)$
then $\mu_X = \sum_k kp_k$.
\end{rem}

\begin{prop}\label{c3p2}
Let $X$ be a discrete random variable and $Y = r(X)$, where $r$ is a bijective
map. Then 
\[
E(Y) = \int r(X)dF_X = \sum_x r(x)f_X(x).
\]
\end{prop}
\begin{proof}
$E(Y) = \sum_y yf_Y(y)$. Since $r$ is bijective, the sum over $y$ can be 
replaced with that over $x$ so that $E(Y) = \sum_x r(x)f_Y(y)$. Now $f_Y(y)
= F_Y(y) - F_Y(y - 1) = P(Y \le y) - P(Y \le y - 1) = P(r(X) \le y) - P(r(X) \le
y - 1) = P(X \le r^{-1}(y)) - P(X \le r^{-1}(y-1)) = F_X(r^{-1}(y)) - F_X(r^{-1}
(y - 1)) = f_X(x)$. Thus, we have
\[
E(Y) = \sum_x r(x)f_X(x).
\]
\end{proof}

\begin{thm}[The Rule of the Lazy Statistician]\label{c3t1}
Let $X$ be a random variable and $Y = r(X)$, then 
\[
\mu_Y = E(r(X)) = \int rdF_X.
\]
\end{thm}
\begin{proof}
\[
E(r(X)) = \int r(x)f_X(x)dx = \int r(x)dF_X.
\]
\end{proof}

If $r(x) = I_A(x)$, the indicator function of the set $A$ then,
\[
E(r(X)) = \int I_A(x)dF_x = \int_A df_X(x) = P(X \in A).
\]

\section{Solutions to problems}
\begin{enumerate}
\item Solved in the accompanying Python notebook.

\item If $P(X = c) = 1$ then $E(X) = c$ and $E(X^2) = c^2$ so that $\var(X) = 0$.
If, on the other hand, $\var(X) = 0$ then $E[(X - E(X))^2] = 0$. Now
$(X - E(X))^2$ is a non-negative random variable. If its expectation is zero then
it should be zero, almost surely. That is $X = E(X)$ almost surely, which in turn
means that $X$ is a constant.

\item Solved in the accompanying Python notebook.
\end{enumerate}

\end{document}