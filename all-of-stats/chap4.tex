\documentclass{article}
\usepackage{amsmath, amssymb, amsfonts, amsthm}
\usepackage{mathtools}
\usepackage{hyperref}
\usepackage{mathabx}

\newcommand{\son}{\mathbb{N}}
\newcommand{\soi}{\mathbb{Z}}
\newcommand{\soq}{\mathbb{Q}}
\newcommand{\sor}{\mathbb{R}}
\newcommand{\soc}{\mathbb{C}}
\newcommand{\op}{\prime}
\newcommand{\tp}{\prime\prime}
\newcommand{\td}[2]{\frac{d{#1}}{d{#2}}}
\newcommand{\bBer}{\text{Bernoulli}}

\theoremstyle{plain}
\newtheorem{thm}{Theorem}
%\numberwithin{thm}{section}

\theoremstyle{plain}
\newtheorem{prop}{Proposition}
%\numberwithin{prop}{section}

\theoremstyle{definition}
\newtheorem{defn}{Definition}
%\numberwithin{defn}{section}

\theoremstyle{remark}
\newtheorem*{rem}{Remark}

\theoremstyle{plain}
\newtheorem{cor}{Corollary}
%\numberwithin{cor}{section}

%\numberwithin{equation}
\begin{document}
\begin{thm}[Markov's inequality]\label{c4t1}
Let $X$ be a non-negative random variable such that $E(X)$ exists. For any 
$t>0$,
\[
P(X \ge t) \le \frac{E(X)}{t}.
\]
\end{thm}
\begin{proof}
We start with the definition,
\[
E(X) = \int_{-\infty}^\infty xdF_X.
\]
We will first consider $X$ to be a continuous random variable. Since $X$ is 
non-negative, $f(x) = 0$ for $x \le 0$. Therefore,
\[
E(X) = \int_0^\infty xf(x)dx = \int_0^t xf(x)dx + \int_t^\infty xf(x)dx \ge
t\int_t^\infty f(x)dx = tP(X \ge t).
\]
If $X$ is discrete,
\[
E(X) = \sum_{x=0}^\infty xf(x) \ge \sum_{x=t}^\infty xf(x) \ge 
t\sum_{x=t}^\infty f(x) = tP(X \ge t)
\]
\end{proof}
\begin{rem}
Note that this inequality is applicable only for non-negative random variables.
Further, the rhs is less than $1$ if $t > E(X)$. For $t < E(X)$, the inequality
is ineffective. By that, I mean that it is true but not useful.
\end{rem}

Consider a geometric random variable with parameter $p$. Its pmf is
\[
f(x) = (1 - p)^x p, x \in \son,
\]
and $E(X) = 1/p$. Let us consider a concrete case with $p = 1/7$. Then, Markov
inequality guarantees that $P(X \ge 4) \le 7/4$, which is true but not useful.
We can easily calculate it as
\[
P(X \ge 4) = \sum_{x \ge 4}(1 - p)^x p = p(1-p)^4 \sum_{x\ge 0}(1-p)^x = 
\frac{p(1-p)^4}{1 - (1 - p)} = (1 - p)^3 = \frac{216}{343}.
\]

\begin{thm}[Chebyshev's inequality]\label{c4t2}
Let $\mu = E(X)$ and $\sigma^2 = \var(X)$. Then,
\[
P(|X - \mu| \ge t) \le \frac{\sigma^2}{t^2}.
\]
\end{thm}
\begin{proof}
Let $Y = (X - \mu)^2$. Then $Y$ is non-negative and its mean is $\sigma^2$. 
Using Markov's inequality,
\[
P(Y \ge t) \le \frac{\sigma^2}{t^2}.
\]
But $P(Y \ge t) = P((X - \mu)^2 \ge t^2) = P(|X - \mu| \ge t)$.
\end{proof}
\begin{rem}
Chebyshev's inequality is useful in estimating the probability of $X$ taking
extreme values on either side of the mean.
\end{rem}

\begin{cor}\label{c4c1}
If $Z = (X - \mu)/\sigma$, where $\mu = E(X)$ and $\sigma^2 = \var(X)$ then
\[
P(|Z| \ge t) \le \frac{1}{t^2}.
\]
\end{cor}
\begin{proof}
Follows from Chebyshev's inequality \ref{c4t2} and $E(Z) = 0, \var(Z) = 1$.
\end{proof}

\begin{thm}[Hoeffding's inequality]\label{c4t3}
Let $Y_1, \ldots, Y_n$ be independent observations such that $E(Y_i) = 0$ and
$a_i \le Y_i \le b_i$, that is $Y_i$ are bounded. For any $\epsilon, t > 0$,
\[
P\left(\sum_{i=1}^n Y_i \ge \epsilon\right) \le e^{-t\epsilon}\prod_{i=1}^n
\exp\left(\frac{t^2(b_i - a_i)^2}{8}\right).
\]
\end{thm}
\begin{proof}
We start with
\[
\sum_{i=1}^n Y_i \ge \epsilon \Rightarrow t\sum_{i=1}^n Y_i \ge t\epsilon,
\] 
for $t > 0$. Since $\exp$ is a monotonically increasing function, we also have
\[
\sum_{i=1}^n Y_i \ge \epsilon \Rightarrow \exp\left(t\sum_{i=1}^n Y_i\right) \ge 
\exp(t\epsilon).
\]
The random variable $Z = \exp\left(t\sum_{i=1}^n Y_i\right)$ takes non-negative
values. Therefore, we can use Markov's inequality $P(Z \ge s) \le E(Z)/s$, where
$s = \exp(t\epsilon)$. Thus,
\begin{equation}\label{e1}
P\left(\sum_{i=1}^n Y_i \ge \epsilon\right) = 
P\left(\exp\left(t\sum_{i=1}^n Y_i\right) \le \exp(t\epsilon)\right) \le 
\frac{E(Z)}{s}.
\end{equation}
Now,
\[
E(Z) = E\left(\exp\left(t\sum_{i=1}^n Y_i\right)\right) = 
\exp\left(t\sum_{i=1}^n E(Y_i)\right)
\]
so that
\begin{equation}\label{e2}
P\left(\sum_{i=1}^n Y_i \ge \epsilon\right) \le 
e^{-t\epsilon}\exp\left(t\sum_{i=1}^n E(Y_i)\right) = 
e^{-t\epsilon}\prod_{i=1}^n E(e^{tY_i}).
\end{equation}
Since $a_i \le Y_i \le b_i$, we can write
\[
Y_i = (1 - \alpha_i)a_i + \alpha_i b_i
\]
where 
\begin{equation}\label{e3}
\alpha_i = \frac{Y_i - a_i}{b_i - a_i}.
\end{equation}
Since $\exp$ is a convex function, $\exp(tY_i) \le (1 - \alpha_i)\exp(ta_i) +
\alpha_i\exp(tb_i)$. Putting the value of $\alpha_i$, we get
\[
\exp(tY_i) \le \frac{b_i - Y_i}{b_i - a_i}e^{ta_i} + 
\frac{Y_i - a_i}{b_i - a_i}e^{tb_i}.
\]
Taking expectation of both sides and noting that $E(Y_i) = 0$,
\begin{equation}\label{e4}
E(e^{tY_i}) \le e^{ta_i}\frac{b_i - a_i e^{t(b_i-a_i)}}{b_i - a_i}.
\end{equation}
If 
\begin{eqnarray*}
\gamma &=& -\frac{a_i}{b_i - a_i} \\
u &=& t(b_i - a_i)
\end{eqnarray*}
then we can write \eqref{e4} as
\begin{equation}\label{e5}
E(e^{tY_i}) \le e^{ta_i}(1 - \gamma + \gamma e^u).
\end{equation}
Since $\gamma u = -ta_i$, we can as well write \eqref{e5} as
\begin{equation}\label{e6}
E(e^{tY_i}) \le e^{-\gamma u}(1 - \gamma + \gamma e^u) = e^{g(u)},
\end{equation}
where
\begin{equation}\label{e7}
g(u) = -\gamma u + \log(1 - \gamma + \gamma e^u).
\end{equation}
We now use the fact that for a smooth function like $g$, there exists $\xi \in
(0, u)$ such that
\[
g(u) = g(0) + ug^\op(0) + \frac{u^2}{2}g^\tp(\xi).
\]
For the function in \eqref{e7}, $g(0) = 0, g^\op(0) = 0$ and hence,
\begin{equation}\label{e8}
g(u) = \frac{u^2}{2}g^\tp(\xi)
\end{equation}
for some $\xi \in (0, u)$. We next show that $g^\tp(u) \le 1/4$. Now, 
\[
g^\tp(u) - \frac{1}{4} = 
\frac{(1 - \gamma)(\gamma e^u)}{(1 - \gamma + \gamma e^u)^2} - \frac{1}{4}
= -\frac{[(1 - \gamma) - \gamma e^u]^2}{4} \le 0
\]
so that equation \eqref{e8} becomes
\begin{equation}\label{e9}
g(u) \le \frac{u^2}{8} = \frac{t^2(b_i - a_i)^2}{8}.
\end{equation}
From \eqref{e6} and \eqref{e9}, we finally get
\[
E(e^{tY_i}) \le  \exp\left(\frac{t^2(b_i - a_i)^2}{8}\right)
\]
Using this in equation \eqref{e2} we get
\begin{equation}\label{e10}
P\left(\sum_{i=1}^n Y_i \ge \epsilon\right) \le e^{-t\epsilon}
\prod_{i=1}^n\exp\left(\frac{t^2(b_i - a_i)^2}{8}\right).
\end{equation}
\end{proof}

Using Hoeffding's inequality \ref{c4t3} we prove that
\begin{thm}\label{c4t4}
If $X_1, \ldots, X_n \sim \dBer(p)$ then for any $\epsilon > 0$,
\[
P(|\bar{X}_n - p| > \epsilon) \le 2e^{-2n\epsilon^2},
\]
where 
\[
\bar{X}_n = \frac{1}{n}\sum_{i=1}^nX_i.
\]
\end{thm}
\begin{proof}
Let $Y = \bar{X}_n - p$. Then $E(Y) = 0$. Further, $0 \le \bar{X}_n \le 1
\Rightarrow -p \le \bar{X}_n - p = Y \le 1 - p$. Therefore, the conditions of
Hoeffding's inequality apply and we have
\[
P(Y \ge \epsilon) \le e^{-t\epsilon}\exp\left(\frac{t^2}{8}\right),
\]
for $t > 0$. Choose $t = 4n\epsilon$ so that
\[
P(Y \ge \epsilon) = P(\bar{X}_n - p) \le e^{-4n\epsilon^2}e^{2n\epsilon^2} =
e^{-2n\epsilon^2}.
\]
We can similarly show that $P(\bar{X}_n-p \le -\epsilon) \le e^{-2n\epsilon^2}$
so that
\[
P(|\bar{X}_n - p| \le \epsilon) \le 2e^{-2n\epsilon^2}.
\]
\end{proof}

\begin{thm}[Mill's inequality]\label{c4t5}
If $Z \sim \dNor(0, 1)$ then
\[
P(|Z| \ge t) \le \frac{2}{\pi}\frac{e^{-t^2/2}}{t}.
\]
\end{thm}
\begin{proof}
$|Z| \ge t \Rightarrow -Z \le -t$ and $Z \ge t$. Since the distribution is 
symmetric, $P(-Z \le -t) = P(Z \ge t)$ and hence $P(|Z| \ge t) = 2P(Z \ge t)$.
Now,
\[
P(Z \ge t) = \frac{1}{\sqrt{2\pi}}\int_t^\infty e^{-z^2/2}dz
\]
so that
\[
tP(Z \ge t) = \frac{t}{\sqrt{2\pi}}\int_t^\infty e^{-z^2/2}dz 
\le \frac{1}{\sqrt{2\pi}}\int_t^\infty ze^{-z^2/2}dz.
\] 
If $u = z^2/2$, $zdz = du$ and the limits of integration go from $t^2/2$ to 
$\infty$, so that
\[
tP(Z \ge t) \le \frac{1}{\sqrt{2\pi}}\int_{t^2/2}^\infty e^{-u}du = 
\frac{1}{\sqrt{2\pi}}e^{-t^2/2}.
\]
That is,
\[
P(Z \ge t) \le \frac{1}{\sqrt{2\pi}}\frac{e^{-t^2/2}}{t}
\]
from which Mill's inequality follows immediately.
\end{proof}

\section{Solutions to problems}
In the accompanying notebook.
\end{document}